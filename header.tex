%#########################################################################
%Dokumentformatierung
%#########################################################################
\PassOptionsToPackage{cmyk}{xcolor} %cmyk is used in the project
\documentclass[paper=a4, fontsize=12pt, fleqn, parskip, oneside, bibliography=totocnumbered, numbers=noenddot,titlepage=true]{scrbook}
% use 'twoside' option if necessary (change if header and footer settings see below)
%#########################################################################

%#########################################################################
% Necessary Packages
%#########################################################################
% language
\usepackage[ngerman,english]{babel}		% cultural typographicals, last language ist main language

% font & symbols
\usepackage[T1]{fontenc}	% utf8: user set encoding
\usepackage[utf8]{inputenc}	% utf8 encoding, needed for 'LaTex internal language'
\usepackage[scaled]{helvet}	% sans-serif font 'Helvetica' for headings (TUM corporate design)
%\renewcommand{\familydefault}{\sfdefault}	% if 'Helvetica' should be font for text also (comment next line '\usepackage{mathptmx}')
\usepackage{mathptmx}		% serif font 'Times' for text (TUM corporate design)
\linespread{1.2}			% set linespread
\usepackage{latexsym}	% provides symbols

% math
\usepackage{amstext}	% math package
\usepackage{amsmath}	% math package
\usepackage{amsxtra}	% math package
\usepackage{amssymb}	% math package
\usepackage[italic]{mathastext}
% 'isomath' sets upper case greek letters italic in accordance with 
% the International Standard ISO 80000-2
\usepackage{isomath}

% tables
\usepackage{longtable}	% tables over multiple pages
\usepackage{colortbl}	% allows to color rows, columns and cells
\usepackage{multicol}	% allows to make multiple columns
\usepackage{multirow}	% allows to make multiple rows
\usepackage{array}		% extends options for colum formats
\usepackage{slashbox}	% tabular cells with diagonal lines
\usepackage{booktabs}	% enhances quality of tables

% graphics and pictures
\usepackage[pdftex]{graphicx}
\DeclareGraphicsExtensions{.pdf}
\usepackage{rotating}	% for rotating figures
\usepackage[final]{pdfpages}	% inlcudes external pdfpages
\usepackage{pdflscape}	% allows rotated pages in pdf-file
\usepackage{float}		% provides the 'H' positioning for floating objects

% other
\usepackage[hang]{footmisc}		% typesetting of footnotes
\usepackage{relsize}	% change od size \larger \smaller ...
\usepackage{enumerate}	% additional argument in enumerate environment
\usepackage{enumitem}	% usercontrol for enumerate, itemize, description
\usepackage{expdlist}	% provides break for itemize, enumerate ...
\usepackage{csquotes}	% used for qotation combined with the babel package
\usepackage{setspace}	% settings for linespaceing
\usepackage{placeins}	% provides \FloatBarrier

%#########################################################################
% number, units and chemical symbols
%#########################################################################
\usepackage{siunitx}
%\sisetup{locale=DE}
\sisetup{detect-all=true}
\usepackage{textcomp}	% für micro in siunitx
\sisetup{
detect-all=true,
locale=UK,
	%per-mode=symbol-or-fraction,	% symbol in text mode and fraction in math mode			
	%output-decimal-marker = {.},	% Komma als Dezimaltrenner
	exponent-product = \cdot,		% Punkt bei exponentiellen Zahlen
	%quotient-mode = fraction,		% Stil: g/mol
	list-units = single,			% Option für \SIlist
	%list-final-separator = ~und~,	% Verbindungswort für \SIlist
	range-units = single,			% Option für \SIrange
	%range-phrase = ~bis~,			% Verbindungswort für \SIrange
	sticky-per = true}				% nur ein \per pro \SI Aufruf

% custom commands
\DeclareSIPrePower\quartic{4}
\DeclareSIPostPower\tothefourth{4}

% custom units
\DeclareSIUnit\torr{Torr}
\DeclareSIUnit\sccm{sccm}
\DeclareSIUnit\au{a.u.}
\DeclareSIUnit\mmHg{mmHg}
\DeclareSIUnit\ppm{ppm}
\DeclareSIUnit\sr{sr}
\DeclareSIUnit\gkat{g_{kat}}
\DeclareSIUnit\wpercent{wt\text{-}\%}

% chemical symbols
\usepackage[version=3,arrows=pgf-filled]{mhchem}

%#########################################################################
% tikz-picture & pgfplots
%#########################################################################
\usepackage{tikz}
\usetikzlibrary{calc,fadings,decorations.pathreplacing,shapes.arrows,arrows,
	chains,matrix,positioning,scopes,spy}

\usepackage{pgfplots}
%\usepgfplotslibrary{external}		% wenn Abbildungen extern gespeichert werden sollen
%\tikzexternalize[prefix=TIKZ/]		% wenn Abbildungen extern gespeichert werden sollen
\usepackage{pgfplotstable}
\pgfplotsset{grid style={dashed}}
\pgfplotsset{compat=newest}
\usepgfplotslibrary{units}

\usepackage{chemfig} % provides the command \chemifg, which draws molecules using the tikz package
\usepackage{tikzscale} % helps scaling tikz pictures with \includegraphics

%#########################################################################
% literature
%#########################################################################
\usepackage[style=chem-angew, backend=biber, bibencoding=utf8, sorting=none]{biblatex} % chem-angew
%\addbibresource{Literature.bib}
\addbibresource{sookbib.bib}
%\addbibresource{D:/.bib}

%#########################################################################
% figure and table caption
%#########################################################################
\usepackage{caption}
\captionsetup{font=small, labelfont=bf, justification=raggedright,skip=10pt, format=plain, width=0.9\textwidth}

%#########################################################################
% subfigure
%#########################################################################
\usepackage{subcaption}	% standard multifigure package 

%#########################################################################
% colors
%#########################################################################
\usepackage[cmyk]{xcolor}

%% TUM-Farben
% Primärfarben
	\definecolor{TUMBlau}{cmyk}{1,0.43,0,0} %Pantone 300
	\definecolor{TUMWeiß}{cmyk}{0,0,0,0} %Weiß
	\definecolor{TUMSchwarz}{cmyk}{0,0,0,1} %Schwarz
% Sekundärfarben
	\colorlet{TUMGrau80}{TUMSchwarz!80} % Grau 80%
	\colorlet{TUMGrau50}{TUMSchwarz!50} % Grau 50%
	\colorlet{TUMGrau20}{TUMSchwarz!20} % Grau 20%
	\definecolor{TUMBlauHell}{cmyk}{1,0.54,0.04,0.19} %Pantone 301
	\colorlet{TUMBlauHell80}{TUMBlauHell!80} %Pantone 301 80%
	\colorlet{TUMBlauHell50}{TUMBlauHell!50} %Pantone 301 50%
	\colorlet{TUMBlauHell20}{TUMBlauHell!20} %Pantone 301 20%
	\definecolor{TUMBlauDunkel}{cmyk}{1,0.57,0.12,0.7} %Pantone 540
	\colorlet{TUMBlauDunkel80}{TUMBlauDunkel!80} %Pantone 540 80%
	\colorlet{TUMBlauDunkel50}{TUMBlauDunkel!50} %Pantone 540 50%
	\colorlet{TUMBlauDunkel20}{TUMBlauDunkel!20} %Pantone 540 20%
% Akzentfarben
	\definecolor{TUMBlauAkzent1}{cmyk}{0.42,0.09,0,0} %Pantone 283
	\definecolor{TUMBlauAkzent2}{cmyk}{0.65,0.19,0.01,0.04} %Pantone 542
	\definecolor{TUMElfenbein}{cmyk}{0.03,0.04,0.14,0.08} %Elfenbein
	\definecolor{TUMOrange}{cmyk}{0,0.65,0.95,0} %Orange
	\definecolor{TUMGruen}{cmyk}{0.35,0,1,0.2} %Grün
% Erweiterte Farbpalette
	\definecolor{TUMMagenta}{cmyk}{0.5,1,0,0.4} %Magenta
	\definecolor{TUMLila}{cmyk}{1,1,0,0.4} %Lila
	\definecolor{TUMCyan}{cmyk}{1,0.03,0.3,0.3} %Cyan
	\definecolor{TUMGruenDunkel}{cmyk}{1,0,1,0.2} %Grün Dunkel
	\definecolor{TUMGruenHell}{cmyk}{0.6,0,1,0.2} %Grün Hell
	\definecolor{TUMGelbHell}{cmyk}{0,0.1,1,0} %Gelb Hell
	\definecolor{TUMGelbDunkel}{cmyk}{0,0.3,1,0} %Gelb Dunkel
	\definecolor{TUMOrangeDunkel}{cmyk}{0,0.8,1,0.1} %Orange Dunkel
	\definecolor{TUMRotHell}{cmyk}{0.1,1,1,0.1} %Rot Hell
	\definecolor{TUMRotDunkel}{cmyk}{0,1,1,0.4} %Rot Dunkel

%#########################################################################
% page margins
%#########################################################################
\usepackage[
%paperwidth=210mm,	% set by a4 option in document class
%paperheight=297mm,	% set by a4 option in document class
top=30mm,
bottom=20mm,
inner=20mm,
outer=20mm,
bindingoffset = 10mm,
includehead=false, % headheight and headsep considered as part of height
includefoot=false,
headheight=10mm, %14mm,
headsep=10mm, %4.4mm,
footskip=10mm, %15mm,
marginparsep=0mm,
marginparwidth=0mm%,showframe
]{geometry}

%
\raggedbottom % nice interspace

% numbering of enumerate environment 1.1.1.1
\renewcommand{\labelenumi}{\arabic{enumi}.)}{\leftmargin0cm \rightmargin1cm}
\renewcommand{\labelenumii}{\arabic{enumi}.\arabic{enumii}.)}
\renewcommand{\labelenumiii}{\arabic{enumi}.\arabic{enumii}.\arabic{enumiii}.)}

%#########################################################################
% header and footer
%#########################################################################
%% one sided
\usepackage[automark,headsepline]{scrlayer-scrpage}	% constructing headers & footers
\automark{chapter}
\ihead[]{} \chead[]{} \ohead[]{\headmark} 
\ifoot[]{} \cfoot[\pagemark]{\pagemark} \ofoot[]{} 

%% two sided
%\usepackage[autooneside=false,automark,headsepline]{scrlayer-scrpage}	% constructing headers & footers
%\ihead[]{} \chead[]{} \ohead[]{\headmark} 
%\ifoot[]{} \cfoot[]{} \ofoot[\pagemark]{\pagemark}


\setkomafont{pageheadfoot}{\normalfont\normalcolor\small}
\pagestyle{scrheadings}

%#############################################################################
% avoiding "Hurenkinder", "Schusterjungen" and wrong hyphenation
%#############################################################################
\pretolerance=250
\tolerance=500
\hyphenpenalty=250
\exhyphenpenalty=100
\hyphenchar\font=\string"7F
\doublehyphendemerits=7500
\finalhyphendemerits=7500
\brokenpenalty=10000
\lefthyphenmin=3
\righthyphenmin=3
\widowpenalty=10000
\clubpenalty=10000
\displaywidowpenalty=10000
\looseness=1

% hyphenation example
\hyphenation{Blumen-topf-erde} %nicht Blumento-Pferde
\hyphenation{So-ur-ce-Co-de}
\hyphenation{Pruf-ungs-aus-schuss}
%#########################################################################
% special characters
%#########################################################################
\newcommand{\Laplace}{\mbox{\setlength{\unitlength}{0.1em}%
                            \begin{picture}(20,10)%
                              \put(2,3){\circle*{4}}%
                              \put(3,3){\line(1,0){13}}%
                              \put(18,3){\circle{4}}%
                            \end{picture}%
                           }%
                     }%
\newcommand{\laplace}{\mbox{\setlength{\unitlength}{0.1em}%
                            \begin{picture}(20,10)%
                              \put(2,3){\circle{4}}%
                              \put(4,3){\line(1,0){13}}%
                              \put(18,3){\circle*{4}}%
                            \end{picture}%
                           }%
                     }%
%#########################################################################
% source code
%#########################################################################
\usepackage{scrhack}	% listings in KOMA-Script
\usepackage{tocbasic}	% specialized tableofcontents                    
\usepackage{listings} 	% Typeset source code listings
\usepackage{xpatch}		% patch to use listings
\newlength{\blub}
\setlength{\blub}{\textwidth - 3pt - 6pt - 3pt + 1pt} % framesep, framexrightmargin, framesep, numbersep
\def\lstbasicfont{\fontfamily{OML}\selectfont} % verbatim-like font, basicstyle was \tt
\def\lststandardfont{\fontfamily{pcr}\fontsize{11pt}{0.7\baselineskip}\selectfont}
\lstset{language=C++, breaklines=true, frame=tb, linewidth=\blub, numbers=right, %
	xleftmargin=0.5em, framesep=3pt,
	frame=single,frameround=tttt, framexrightmargin=6pt, numbersep=-1pt,
	basicstyle=\lststandardfont, showstringspaces=false, numberstyle=\tiny,
	commentstyle=\color{TUMGrau50}, escapeinside={(*@}{@*)}, keywords={}
	}
\renewcommand\lstlistingname{Sourcecode}
\renewcommand\lstlistlistingname{List of Source Codes} %List of Source-Codes

%######################################################################### 
% hyperref code
%#########################################################################
\PassOptionsToPackage{hyphens}{url}	% option for line break in urls

\usepackage[pdftex,hyperindex=true, plainpages=false,
	pdfpagelabels,pdfborderstyle={/S/U/W 1},bookmarksnumbered]{hyperref}

\makeatletter	% line breaks in urls
\g@addto@macro\UrlBreaks{
	\do\a\do\b\do\c\do\d\do\e\do\f\do\g\do\h\do\i\do\j
	\do\k\do\l\do\m\do\n\do\o\do\p\do\q\do\r\do\s\do\t
	\do\u\do\v\do\w\do\x\do\y\do\z\do\&\do\1\do\2\do\3
	\do\4\do\5\do\6\do\7\do\8\do\9\do\0}
\makeatother
%#########################################################################
%%EOF
%
\frenchspacing